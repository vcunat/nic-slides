% vim:set foldmethod=expr foldexpr=getline(v:lnum)=~'^\\\\slide.*'?'>1'\:'=' :
% the modeline ^^ apparently doesn't work for some reason
\input nicslix
\englishenv

\pageAspectRatio[16:9]

%\title{Knot Resolver}
\subtitle{\bigskip\pdfpic[1000]{logo-kresd.pdf}}
\author{Vladimir Cunat}{vladimir.cunat@nic.cz}
	% "Č" doesn't render in this font :-(
\date[2018-05-23]


\slide[Projects @ cz.nic]
  % TODO: at beginning or the end?
Czech ccTLD registry,
but many other projects, e.g.:
  % First a few words about my association;
  % cz.nic is primarily our top-level domain registry,
  % but we do lots of other projects - usually it SW
  % and almost always open-source related to the foundations
  % of the internet.
\bulletList
\item FRED: registry SW, used by several ccTLDs
  % We develop the registry SW we use for .cz,
  % and it's _also_ used by several other TLDs.
  % Well, that probably won't be interesting for you,
  % so let me move on.
\item Turris: self-updating secure hackable routers
  % Turris is a family of routers - the main idea here is that
  % all SW is buggy, so old firmware is unlikely to remain secure.
  % Therefore, tey are self-updating (by default).
\item BIRD: routing daemon
\item CSIRT.CZ: national security response team
\item HaaS: honeypot as a service (ssh, telnet)
  % Everyone might run their own honeypot, but simulating ssh
  % is potentially risky, so we offer proxying the connections
  % to us for analysis.  A large diverse set of honeypot IPs
  % is more difficult to avoid.
  % HaaS is connected to Turris - the IPs that are detected
  % as malicious (or infected) get blocked in Turris firewalls.
\item Knot DNS: authoritative-only server
  % Finally our older sister project Knot-DNS,
  % a very fast authoritative server.
\endBulletList


\slide[Contents]
Knot resolver is caching validating DNS resolver.

\bulletList
\item introduction and architecture
  % I'll start with some general and superficial information
  % about the project.
\item overview of features
  % then some overview of what it can do for you;
\item configuration, filtering DNS
  % and finally we'll dive deeper into configuration,
  % in particular into possibilities around DNS filtering.
\endBulletList


\slide[Knot resolver -- in a few lines]
%TODO title?? + express it's about kresd (and not e.g. cz.nic)
\bulletList

\item DNSSEC and privacy
  % cz.nic believes in DNSSEC, so we try to make it just work
  % as easily as possible.

\item flexibility, via modules and lua config
  % Flexibility -- not by providing a thousand configuration options,
  % but rather providing places where you can plug-in lua scripts
  % -- our modules or your own code.

\item minimalistic approach -- libraries
  % We try to defer code to libraries where possible.
  % It's not _just_ about being lazy.

\item simplicity by default
  % You can just start it without any configuration or parameters.
\endBulletList


\slide[Architecture]
Written in C and Lua. \par
No threads, just processes sharing cache.

\bulletList
\item LMDB -- cache, persistent and shareable
\item Knot-DNS libraries -- DNS wire-format, zone files, etc.
\item GnuTLS -- TLS and other crypto
\item libuv -- network IO, timers, other asynchronous stuff
\item LuaJIT -- configuration, scripting (e.g. modules)
\endBulletList


\slide[Features: DNSSEC algorithms]
\image[800]{rootcanary.png}
%\URL{https://rootcanary.org/test.html}
  % Just as other popular resolvers, we support 
  % On the left there are three algorithms that are now considered insecure.
  % We're basically only limited by GnuTLS:
  %  - GOST has a merge request waiting,
  %  - ed448 doesn't

\slide[Features: privacy (1)]
\bulletList

\item query name minimization, by default (rfc7816)
  % For example, the root servers will only see the top-level name (e.g. "com")
  % and not the whole query.
  % Mention empty non-terminal problems?

\item DNS over TLS (rfc7858)
	\dashList
	\item as client: forwarding, trust via key pin or cert
	  % Kresd can forward over TLS - the connection is shared by all requests;
	  % queries and answers can flow out of order.

	\item as server: uses ephemeral cert, by default;
		with padding (rfc7830)
		  % ATM we don't pad when forwarding over TLS.
	\endDashList

\endBulletList
\slide[Features: privacy (2)]
\bulletList

\item aggressive caching (rfc8198)
	\dashList
	\item NSEC zones now, NSEC3 within weeks
	\par (useless in opt-out zones)
	\item also helps against random subdomain attacks
	\endDashList

\item prefill module
	\dashList
	\item whole root zone kept fresh in cache
	\par (alternative to rfc7706: root on loopback)
	\endDashList

%\item rfc6761: % special-use zones, doesn't seem significant enough
\endBulletList


\slide[Features: others]
Provided by modules and off by default:
\bulletList
\item collecting and sending statistics (e.g. to InfluxDB)
\item prefetching records, with prediction
\item serving expired records, if upstream doesn't respond repeatedly
\item DNS64 (rfc6147)
\item query policies, views, ACLs (details later)
\item ...
\endBulletList


\slide[Configuration]
\bulletList
\item flexibility: it's just lua (LuaJIT)
  % Having a full programming language as the configuration language
  % has advantages: you don't need to generate configuration files
  % - you may just put the code _inside_ them.
\item Knot resolver -specific utility functions and bindings
\item control via a lua read-eval-print loop (unix socket)
\item {\tti example 1}
\endBulletList



\slide[Policy module: basics]
\bulletList
\item incoming request is matched against a list of \cznicred filter\endCol{} rules
\item the first matching rule selects an \cznicred action\endCol
\endBulletList

{\bf Available filters:}
\bulletList
\item {\tt all(action)}, {\tt pattern(action, regexp)}, {\tt suffix(action, table)}
\item {\tt rpz(default\_action, path)} -- a subset of the RPZ (more later)
\item custom rule in lua (worse docs, but code is short)
\endBulletList

\slide[Policy module: actions]
\bulletList
\item {\tt DENY}, {\tt DENY\_MSG(message)} -- NXDOMAIN, message in ADDITIONAL TXT
\item {\tt DROP} -- answer SERVFAIL
\item {\tt PASS} -- useful as an exception from further rules
\item {\tt FORWARD(ip)}, {\tt TLS\_FORWARD(ip, auth. params)} -- solve via forwarding
\item {\tt MIRROR(ip)}, {\tt REROUTE({{subnet,target}, ...})}
\item ...
\item {\tti example 2:} filters with actions
\endBulletList

\slide[Policy module: RPZ subset]
\bulletList
\item config: {\tt policy.rpz(action, path)} \par
\item RPZ lines: {\tt *.matched.subtree.com CNAME right.hand.rule.}
\endBulletList
Supported right-hand side rules:
\bulletList
\item * and .* -- the {\tt action} is used (i.e. partial support)
\item {\tt rpz-passthru.} -- as {\tt policy.PASS}
\item {\tt rpz-drop.} -- as {\tt policy.DROP}
\item {\tt rpz-tcp-only.} -- as {\tt policy.TC}
\endBulletList


\slide[View module]
Another layer around policy-filter rules,
based on \cznicred who \endCol asks:
\bulletList
\item {\tt addr(subnet, policy)} -- matching by subnet
\item {\tt tsig(key, policy)} -- matching by TSIG key
\endBulletList
\par
Example:
\verbatim
modules = { 'view', 'policy' }
view:addr('192.168.1.0/24', policy.rpz(policy.PASS, 'whitelist.rpz'))
§



\lastSlide[\cznicred https://knot-resolver.cz\endCol]{Thanks! Questions?}
\bye


\slide
DNSSEC:
\item trust anchors: updating (rfc5011), negative (rfc7646), signalling knowledge (rfc8145)

- slight IPv6 preference

- TTY config (runtime reconfig.)

https://www.ctrl.blog/entry/kresd-random-dns-forwarding
https://forum.turris.cz/t/knot-resolver-and-rpz/2023


