% vim:set foldmethod=expr foldexpr=getline(v:lnum)=~'^\\\\slide.*'?'>1'\:'=' :
% the modeline ^^ apparently doesn't work for some reason
\input nicslix
\englishenv

\pageAspectRatio[16:9]

%\title{Knot Resolver}
\subtitle{\bigskip\pdfpic[1000]{logo-kresd.pdf}}
\author{Vladimir Cunat}{vladimir.cunat@nic.cz}
	% "Č" doesn't render in this font :-(
\date[2018-05-23]


\slide[Projects @ cz.nic]
  % TODO: at beginning or the end?
Czech ccTLD registry,
but many other projects, e.g.:
\bulletList
\item FRED: registry SW, used by several ccTLDs
\item Turris: self-updating secure hackable routers
\item BIRD: routing daemon
\item CSIRT.CZ: national security response team
\item HaaS: honeypot as a service
\item Knot DNS: authoritative-only server
  % Finally our older sister project Knot-DNS,
  % a very fast authoritative server.
\endBulletList


\slide[Contents]
Knot resolver is caching validating DNS resolver.

\bulletList
\item introduction and architecture

\item overview of features

\item configuration, filtering DNS

\endBulletList


\slide[Knot resolver core values]
%TODO title?? + express it's about kresd (and not e.g. cz.nic)
\bulletList

\item minimalistic approach -- libraries
  % We try to defer code to libraries where possible.
  % It's not _just_ about being lazy.

\item simplicity by default
  % You can just start it without any configuration or parameters.

\item DNSSEC and privacy
  % cz.nic believes in DNSSEC, so we try to make it just work
  % as easily as possible.

\item flexibility, via modules and lua config
  % Flexibility -- not by providing a thousand configuration options,
  % but rather providing places where you can plug-in lua scripts
  % -- our modules or your own code.
\endBulletList


\slide[Architecture]
No threads, just processes sharing cache.

\bulletList
\item LMDB -- cache, persistent and shareable
\item Knot-DNS libraries -- DNS wire-format, zone files, etc.
\item GnuTLS -- TLS and other crypto
\item libuv -- network IO, timers, other asynchronous stuff
\item LuaJIT -- configuration, scripting (e.g. modules)
\endBulletList


\slide[Features: DNSSEC algorithms]
\image[800]{rootcanary.png}
%\URL{https://rootcanary.org/test.html}
  % On the left there are three algorithms that are now considered insecure.
  % We're basically only limited by GnuTLS:
  %  - GOST has a merge request waiting,
  %  - ed448 doesn't

\slide[Features: privacy (1)]
\bulletList

\item query name minimization, by default (rfc7816)
  % For example, the root servers will only see the top-level name (e.g. "com")
  % and not the whole query.
  % Mention empty non-terminal problems?

\item DNS over TLS (rfc7858)
	\dashList
	\item as client: forwarding, trust via key pin or cert
	  % Kresd can forward over TLS - the connection is shared by all requests;
	  % queries and answers can flow out of order.

	\item as server: uses ephemeral cert, by default;
		with padding (rfc7830)
		  % ATM we don't pad when forwarding over TLS.
	\endDashList

\endBulletList
\slide[Features: privacy (2)]
\bulletList

\item aggressive caching (rfc8198)
	\dashList
	\item NSEC zones now, NSEC3 within weeks
	\par (useless in opt-out zones)
	\item also helps against random subdomain attacks
	\endDashList

\item prefill module
	\dashList
	\item whole root zone kept fresh in cache
	\par (alternative to rfc7706: root on loopback)
	\endDashList

%\item rfc6761: % special-use zones, doesn't seem significant enough
\endBulletList


\slide[Features: others]
Provided by modules and off by default:
\bulletList
\item collecting and sending statistics (e.g. to InfluxDB)
\item prefetching records, with prediction
\item serving expired records, if upstream doesn't respond repeatedly
\item DNS64 (rfc6147)
\item query policies, views, ACLs (details later)
\item ...
\endBulletList


\slide[Configuration]
\bulletList
\item flexibility: it's just lua (LuaJIT)
\item Knot resolver -specific syntax sugar and bindings
\item control via a unix socket -- lua read-eval-print loop
\item {\tti example 1}
\endBulletList


\slide
DNSSEC:
\item trust anchors: updating (rfc5011), negative (rfc7646), signalling knowledge (rfc8145)

- slight IPv6 preference

- TTY config (runtime reconfig.)

Filtering:
- policy mod.
- views mod.

https://www.ctrl.blog/entry/kresd-random-dns-forwarding
https://forum.turris.cz/t/knot-resolver-and-rpz/2023


\lastSlide[https://knot-resolver.cz]{Thanks! Questions?}

\bye
