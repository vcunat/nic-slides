% vim:set foldmethod=expr foldexpr=getline(v:lnum)=~'^\\\\slide.*'?'>1'\:'=' :
% the modeline ^^ apparently doesn't work for some reason
\input nicslix
\englishenv

\pageAspectRatio[16:9]

\title{DNS Flag Day 2020\hfill\pdfpic[300]{DNS_Flag.pdf}}
\author{Vladimir Cunat}{vladimir.cunat@nic.cz}
	% "Č" doesn't render in this font :-(
\date[2019-11-26]

\def\newLine{\hfill\break}
\def\textSoft#1{\startCol{0.67 0.81 1   }#1\endCol}
	%\cznicblue{\startCol{0    0.22 0.57}}


\slide[Intro: Flag Day]
% Just shortly.
Coordinated action: open-source DNS vendors + operators
% mainly because the first to move is disadvantaged
{\newLine\it\textSoft{\hfill
	> It's only broken with your resolver, so it's obviously your fault.}

\URL{https://dnsflagday.net}

{\bf 2019, February:}
\bulletList
\item no reaction to a query $\to$ do not retry without EDNS
\item still allowed not to support EDNS: reply {\tt FORMERR}, RFC 6891
\bigskip
\item no real incident happened
\endBulletList


\slide[2020: IP fragmentation]
"Smaller" than 2019 \par
Aim: avoid fragmented UDP packets (with DNS) \par
Why:
\bulletList
\item{\bf reachability}: large UDP often can't pass through \textSoft{\hfill(especially on IPv6)}
\item{\bf security}: fragmentation simplifies off-path attacks on unsigned data
\endBulletList
% add an image?


\slide[Plan]
Plan: limit DNS payload over UDP by default, to 1232 bytes \newLine
	\textSoft{\hfill(by both sides, can be overridden if you know better)}

Unchanged protocol:
\bulletList
\item EDNS off $\to$ limit = 512
\item EDNS on  $\to$ client chooses an upper bound \textSoft{(and server may also trim that)}
\item current standard recommends 4096 limit as default
\item answer does not fit $\to$ {\tt TC=1}, retry over TCP
\endBulletList


\slide[Potential issues]
\bulletList
\item server does not honor client's choice of upper bound
% Now various variants of not noticing that TCP is *mandatory* part of DNS.
\item server does not listen on TCP/53 \textSoft{(clarified in RFC 7766)}
\item TCP/53 gets blocked (e.g. firewalls)
\item client does not use TCP on an answer with TC=1
\item \textSoft{TCP latency (edge cases)}
\endBulletList

\medskip\dots but
\bulletList
\item millions of users switched already \textSoft{(without issues)}
\endBulletList


\slide[ToDo]
\bulletList
\item no date yet, probably second half of 2020
\item testing in advance: config is easy + web tool provided
\item contribute: translations, "supporters" list, \dots
\item zone reccomendation: avoid needing large replies \textSoft{(e.g. ECDSA helps)}
\endBulletList

\vfill
\cznicred
Questions?  Feedback?
\cznicblue



\slide[Links]
\URL{https://dnsflagday.net/2020/}

\def\softURL#1{\petit\newLine{\URL[x]{#1}}\normalSize}
\bulletList
\item IP Fragmentation Considered Fragile \textSoft{(BCP track, now in RFC Ed Queue)}
	\softURL{https://tools.ietf.org/html/draft-ietf-intarea-frag-fragile}
\item IPv6, Large UDP Packets and the DNS \textSoft{(article by Geoff Huston)}
	\softURL{http://www.potaroo.net/ispcol/2017-08/xtn-hdrs.html}
\item Measures against cache poisoning attacks using IP fragmentation in DNS
	\softURL{https://tools.ietf.org/html/draft-fujiwara-dnsop-fragment-attack}
	\hfill\hbox{\textSoft{(I-D, expired)}}
\endBulletList


\slide[Configuration]

\twoColumn{0.5}{0.5}
BIND
\petit \verbatim
options {
  edns-udp-size 1232;
  max-udp-size 1232;
};
§ \normalSize \smallskip

NSD
\petit \verbatim
server:
  ipv4-edns-size: 1232
  ipv6-edns-size: 1232
§ \normalSize \smallskip

Unbound
\petit \verbatim
server:
  edns-buffer-size: 1232
§ \normalSize \smallskip

\rightColumn
Knot DNS
\petit \verbatim
server:
  max-udp-payload: 1232
§ \normalSize \smallskip

Knot Resolver
\petit \verbatim
net.bufsize(1232)
§ \normalSize \smallskip

PowerDNS Authoritative
\petit\verbatim
udp-truncation-threshold=1232
§ \normalSize \smallskip

PowerDNS Recursor
\petit\verbatim
edns-outgoing-bufsize=1232
udp-truncation-threshold=1232
§ \normalSize \smallskip

\endColumn

\bye
