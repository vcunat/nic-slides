% vim:set foldmethod=expr foldexpr=getline(v:lnum)=~'^\\\\slide.*'?'>1'\:'=' :
% the modeline ^^ apparently doesn't work for some reason
\input nicslix
\czechenv

\pageAspectRatio[16:9]

\title{Rychlejší packety díky XDP}
\subtitle{a předběžné zkušenosti v Knot Resolveru}
\author{Vladimír Čunát}{vladimir.cunat@nic.cz}
\date[2019-10-01]


\slide[Přehled]

\bulletList
\item motivace: rychlost
\item (e)BPF: úvod, příklad
\item XDP a AF\_XDP
\endBulletList


\slide[Motivace]

relativně drahé síťové syscally -- profil CPU času:
\bulletList
\item Knot DNS: 44\% sendmmsg, 14\% recvmmsg
\item Knot Resolver: 33\% sendmmsg, 14\% recvmsg
\item Unbound: 39\% sendto, 15\% recvfrom
\endBulletList
(UDP, odpovědi z cache, fyzická síť)


\slide[(e)BPF]
\twoColumn{0.66}{}
(extended) Berkeley Packet Filter
\bulletList
\item původně: filtrování packetů
\item kód od uživatele spuštěný už v kernel-módu
\item instrukční sada pro velmi omezený virtuální stroj

(C $\to$ clang)
\item verifikace že kód nemůže udělat nic špatného
\item dnes (Linux): více funkcí, využití i jinde (tracing)
\endBulletList

\rightColumn
\bigskip
\image[150]{pony_ebpf01.png}
\endColumn


\slide[XDP]

eXpress Data Path (Linux 4.8)
\bulletList
\item eBPF před síťovým stackem
\item může packet upravit
\endBulletList

nakonec rozhodnutí:
\bulletList
\item zahodit, zpracovat normálně
\item odeslat ze stejného nebo jiného interface
\item přesměrovat přímo do user-space AF\_XDP socketu (Linux 4.18)
\endBulletList

%% zde příklad: dns-reflector?  Ale vlastně v něm nic moc záživného není.



\bye

