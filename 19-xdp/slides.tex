% vim:set foldmethod=expr foldexpr=getline(v:lnum)=~'^\\\\slide.*'?'>1'\:'=' :
% the modeline ^^ apparently doesn't work for some reason
\input nicslix
\czechenv

\pageAspectRatio[16:9]

\title{Rychlejší packety díky XDP}
\subtitle{a předběžné zkušenosti \hfill\pdfpic[400]{logo-kresd.pdf}}
\author{Vláďa Čunát}{vladimir.cunat@nic.cz}
\date[2019-10-01]


\slide[Přehled]

%FIXME
\bulletList
\item Motivace: rychlost
\item (e)BPF: úvod
\item XDP: lepší (e)BPF filtrování
\item AF\_XDP: socket, odesílání, příjem
\item AF\_XDP: extras, nevýhody
\item Naše plány
\endBulletList


\slide[Motivace: rychlost]

\twoColumn{0.66}{0.5}
relativně drahé síťové syscally -- profil CPU času:
\bulletList
\item Knot Resolver: 33\% sendmmsg, 14\% recvmsg
\item Knot DNS: 44\% sendmmsg, 14\% recvmmsg
\item Unbound: 39\% sendto, 15\% recvfrom
\endBulletList
(UDP, odpovědi z cache, fyzická síť)

\rightColumn
\bigskip
\hfill{}profil Knot DNS\par
\hfill\pdfpic[150]{flame-graph.pdf}
\endColumn


\slide[(e)BPF: extended Berkeley Packet Filter]

\twoColumn{0.66}{0.5}
\bulletList
\item původně: filtrování packetů v BSD (1992 článek)
\item kód od uživatele spuštěný už v kernel-módu
\item instrukční sada pro velmi omezený virtuální stroj
C $\to$ clang $\to$ bytecode $\to$ kernel
\item verifikace že kód nemůže udělat nic špatného
\item dnes (Linux): více funkcí, využití i jinde (tracing)
\endBulletList

\rightColumn
\bigskip
\hfill\image[150]{pony_ebpf01.png}\par
\hfill{}neoficiální logo eBPF
\endColumn


\slide[XDP: lepší (e)BPF filtrování]

\bulletList
eXpress Data Path (Linux 4.8, rok 2016)
\bulletList
\item eBPF před síťovým stackem
\item může packet upravit (ale zatím ne prodloužit)
\endBulletList

nakonec rozhodnutí, např.:
\bulletList
\item zahodit, zpracovat normálně
\item odeslat ze stejného nebo jiného interface
\item přesměrovat přímo do user-space AF\_XDP socketu (Linux 4.18, rok 2018)
\endBulletList
% TODO zde příklad: dns-reflector?  Ale vlastně v něm nic moc záživného není.


\slide[AF\_XDP: socket]

\twoColumn{0.75}{0.5}
\bulletList
\item obchází OS stack s vybranými ether.~framy
\item navázán na jednu frontu jednoho net-interface
\item RX i TX, důraz na minimalizaci overheadů
\item využití sdílení paměti mezi kernel-space a user-space
\item umem: předalokované pole bufferů stejné velikosti
\item 4 kruhové fronty pro posílání deskriptorů ($\to$ umem)
%\item následná komunikace (téměř) bez syscallů, ale
\item lze použít obvyklé (e)poll/select
\endBulletList

\rightColumn
\smallskip
\hfill Data $\ne$ Diesel\par
\hfill\image[175]{other-xdp.jpg}\par
\endColumn


\slide[AF\_XDP: odesílání]

\twoColumn{0.75}{0.53}
\bulletList
\item připravit frame do umem bufferu (někam)
\item deskriptor $\to$ TX fronta (adresa a délka framu)
\item probudit ovladač syscallem\par
	($\le$ 2019, stačí jednou na dávku)
\item ovladač předá buffer kartě
\item deskriptor $\to$ "completion" fronta
\endBulletList

\rightColumn
\bigskip\bigskip\bigskip\smallskip
\hfill\image[600]{lwn-af_xdp.png}\par
\endColumn


\slide[AF\_XDP: příjem]

\twoColumn{0.5}{0.53}
\bulletList
\item deskriptory $\to$ "fill" fronta
\item packet který projde BPF je zapsán
\item a jeho deskriptor $\to$ RX fronta
\endBulletList
detaily:
\bulletList
\item průběžně doplňovat "fill" frontu
\item lze nastavit padding před framy
	use case: přidání hlaviček in-place
\endBulletList

\rightColumn
\bigskip\bigskip\bigskip\smallskip
\hfill\image[600]{lwn-af_xdp.png}\par
\endColumn


\slide[AF\_XDP: extras]

data se nezkopírují ani jednou, ideálně:
\bulletList
\item Intel (i40e, ixgbe): Linux 4.20 (Vánoce 2018)
\item Mellanox (mlx5): Linux 5.3 (před dvěma týdny)
\endBulletList

vylepšení v budoucnu:
\bulletList
\item komunikace bez syscallů (i při odesílání)
\item ještě větší flexibilita v (e)BPF programech
\item snad efektivnější duplikace packetů
\endBulletList


\slide[AF\_XDP: nevýhody]

\twoColumn{0.7}{0.5}
cena TX klesne téměř na nulu, ale:
\bulletList
\item obejití OS stacku má nevýhody
	\bulletList
	\item bez čehokoliv v netfilteru
	\item zatím: symetrické routování
	\endBulletList
\item složitější nastavování
\item práva: skoro root (CAP\_SYS\_ADMIN, \dots)
\item výrazné zrychlení jen na některých kartách
\endBulletList

\rightColumn
\smallskip
\pdfpic[300]{resolver-xdp.pdf}
\par\hfill Knot Resolver: TX s XDP
\endColumn


\slide[Naše plány]

\twoColumn{0.9}{}
nevýhody $\to$ default bez XDP, ale speciální případy
\bulletList
\item triviální nástroje: dns-reflector (UDP)
\item dokončení integrace do Knot Resolveru (UDP $\leftarrow$ klienti)
\item integrace do Knot DNS (UDP)
\item \URL{https://knot-dns.cz/benchmark/}
\endBulletList
\medskip
\bulletList
\item jiné nápady?
\par statistky (dns-collector?), load-balancing, \dots
\endBulletList

\rightColumn
\bigskip\bigskip
\image[2000]{Comcast-Logo.png}
\endColumn


\slide[Užitečné odkazy]

\bulletList
\item \URL{https://lwn.net/Articles/750845/}
\item \URL{https://www.kernel.org/doc/html/latest/networking/af\_xdp.html}
\item \URL{https://archive.fosdem.org/2018/schedule/event/af\_xdp/}
\item \URL{https://github.com/xdp-project/xdp-tutorial}
\endBulletList
(ty s podtržítkem nejsou klikatelné, bohužel)

\bye

